\documentclass[letterpaper]{article}
\usepackage{url}
\usepackage[margin=1in]{geometry}
\newcommand{\breview}{\begin{quotation}\begin{bf}\noindent}
\newcommand{\ereview}{\end{bf}\end{quotation}}

\begin{document}

\section*{Reviewer \#1}

\breview

The authors give a good overview of the existing methods to predict
the retention time.  However, the description of the methods is
sometimes too brief, and slightly biased towards their own developed
methods. Extending the descriptions of the methods that are not
developed by the authors should be considered. The applications and
discussion section are an added value to the paper.

\ereview

We would like to thank the reviewer for the positive feedback. We have
revised the sections were the other methods contribution is discussed


\breview
Additional comments:
\begin{itemize}
\item Legend of Figure 2 overlaps with text on page 3
\item Table 1 is not fully visible; additionally in the table ``Elude'' is written, while in the text ``ELUDE'' is used
\end{itemize}
\ereview
We have corrected this.
\breview
\begin{itemize}

\item The authors write on page 5 and 6: ``The algorithm reported good
accuracy, especially taking into account that it only used information
about the amino acid composition of the peptides.'' This statement
seems to be contrasting with the following comment found on page 3:
``As a consequence, given a chromatography setup, the sequence of a
peptide should provide virtually all the information needed to predict
its retention time''. To the latter statement several other
contrasting statements can be found on the same page.
\end{itemize}
\ereview
We added a qualifying paragraph to the later statement.
``As a consequence, given a chromatography setup, the sequence of a
peptide should provide virtually all the information needed to predict
its retention time. However, in practice this is not as straight
forward as it might seem, as it for instance is difficult to predict a
peptide's three dimensional structure from its amino acid sequence.''
\breview
\begin{itemize}
\item The paragraph in Section 4 on the modified ELUDE is not completely clear. An extended discussion would be useful
\end{itemize}
\ereview

We revised this section to better explain how we incorporated
retention time predictions of modified peptides. 

\breview
Typos:
\begin{itemize}
\item Page 6: Klammer et al. $\to$ Klammer {\em et al.}
\item Page 6: Moruz et al.[22] $\to$ Moruz {\em et al.} [22]
\item Page 7: Strittmatter et al. $\to$ Strittmatter {\em et al.}
\item Page 7: Klammer et al. $\to$ Klammer {\em et al.}
\item Page 8: Krokhin et al. $\to$ Krokhin {\em et al.}
\item Page 8: bigdata $\to$ big data
\item Page 9: retention time[57] $\to$ retention time [57].
\item Page 9: DIA $\to$ data independent acquisition
\end{itemize}

\ereview

Thank you for highlighting these typos. We have now corrected them.


\section*{Reviewer \#2}

\breview


The reviewer feels the paper’s target audience isn’t clear. As such,
the overall sensation is that the paper touches several subjects
lightly but fails in going in detail into any of them. As it is right
now, this paper is not practical enough for a mass spectrometrist, and
not technical enough for a machine learning scientist. The reviewer
feels it is necessary to define an audience and focus the paper to
this audience’s expertise.

Specific Comments:

As it was mentioned before, the main concern of the reviewer is the
lack of focus in the paper. As such the main point of the review is to
clarify who is the audience the paper is meant for and delve deeper
into what is relevant for them.

\ereview

We feel that while the reviewer is right in that this manuscript
targets more than one audience, and that we due to brevity reasons
have to skip some details. However, we also feel that this is the
point of a review, to give an overview for whoever wants to get an
introduction to a subject. The details can be found in the
publications we refer to. Regardless, we updated the text to better
fit into a category of reader that we would define as a mass
spectrometrist with a fair understanding of computational methods that
is interested in knowing were the field of retention time predictions
stands today.  

\breview


Other than that, there are some points worth mentioning. They will be
here presented in the order they appear on the manuscript.

On section 2, it is written ``so-called linear gradients''. Why say
``so-called''? This expression gives the idea that the gradients in
reality are not linear, but this isn’t further developed.

\ereview

The selection of quotes might in part due to the strange nomenclature
in the field of chromatography. In other fields a gradient refers to
the vector of spacial partial derivates, not a time derivate.

However, to avoid confusion, we replaced the word so-called with
setting the words linear gradient in italics, to indicate that a
explanation of the term will follow. 

\breview

The reproducibility of RPLC is a big issue. Not only does a column get
degraded with use, but there is also documented variability in
products from different manufacturers. In this section it is also
noted how the retention time depends on several experimental settings
– ``variety of parameters such as column dimensions, the composition
of stationary and mobile phase, temperature or gradient slope''. So we
have several factors independent of the peptide sequence that
influence greatly the retention time of a given peptide. In the
following section it is said that ``given a chromatography setup, the
sequence of a peptide should provide virtually all the information
needed to predict its retention time''; however, given that there is
so much variability in the chromatography setup in different labs and
different experiments, we feel that it should be emphasized how big of
an issue this is in retention time prediction and note efforts made in
addressing this.

\ereview
We added a qualifying section to this section.

``However, in practice such predictive models is not as straight
forward as it might seem. For instance, one important part of such a
model would be the peptides three dimensional conformation. Although
the current structure prediction methods are quite accurate in
predicting three dimensional conformation of short amino acid chains,
such methods will not be equipped with the flexibility to model the
various conditions of a chromatographic experiment. For instance, many
peptides are likely changing their confirmation as a function of the
amount of organic solvent in the mobile phase, an effect that is
likely to have an important contribution to each peptides retention
time. Hence, most efforts to predict retention time have moved into
treating the chromatographic process as a black box, that is a system
that is only viewed in terms of its inputs an outputs.''


\breview

There are some issues with the caption for Fig. 2. It is also not
clear if there are points that fall on the regression line. 

\ereview

We changed the formating of the manuscript so that the caption of
Figure2 should stand clear. There is no regression line in Fig. 2.

\breview

Besides, this figure is not representative of a complex sample or of
the reproducibility (or variability) in experiments between different
labs, or even different machines. As such the figure can be misleading
in terms of evaluating the reproducibility of a shotgun proteomics
experiment.

\ereview

We added a qualifying sentence to the caption of the Figure: ``It
should be noted that this relatively high reproducibility normally
only appears between runs under near to identical experiments, and do
not carry over into other experimental conditions or instrumentation.''

\breview

Across section 3, there is a lack of evaluation of the prediction
methods. In fact at a certain point it is said that one is ``good'',
one ``excellent'', and another that has ``lower'' accuracy followed by
one that is ``reasonable''. It is inferred that if these qualitative
comparisons are made by the authors, then there must be some accuracy
measure available. It would be preferable to have a table with these
values. 

\ereview

We removed all these unqualified references to methods accuracy.  

\breview

Table 1 falls out of the page.
\ereview
We modified the formating of the table.
\breview

In section 3.1, the last paragraph doesn’t seem relevant to the
section it is included in. Maybe it would make more sense to include
it in the applications section.

\ereview

We did not feel that the paragraph sat better in the application
section and letted the sentence remain as is.

\breview

In section 3.2, it feels relevant to introduce the approach ELUDE
takes regarding retention indexes. Even if they obtained via, are used
as features for, a machine learning approach, it is still relevant to
this section.  

\ereview

We added a couple of sentences describing this.

\breview

Section 3.3 is rather brief. The reviewer feels there should be more
information on these methods (or none at all).

\ereview

We added some more details on how BioLCCC works to the text.

\breview

On sections 3.1, 3.2 and 3.3 there is no evaluation of any of the
introduced methods. In section 3.4 we see the sort of qualitative
comparisons mentioned above. We note that there is no mention of how
the performance of ELUDE compares to the other methods, even if
qualitatively.

\ereview
We added a reference to a benchmark to the section.
\breview

If the proposed audience to this paper are scientists applying machine
learning methods to retention time prediction, more information on the
features used and the methods themselves would be relevant. 

\ereview

We included an brief enumeration of some of the features used in Elude.

\breview

In the discussion the reviewer feels that it should be discussed not
only how retention time prediction can be improved, but also the
limitations of how accurate predictions can be, given the issues
mentioned above. Another relevant question is how much more relevant
do these predictions need be given the applications enumerated in
section 5.

\ereview

Estimate of how good predictions should be to be useful are generally
hard to make. Better performance will always be more useful than worse
performance.  Instead we added a comparison of the FPR of retention
time predictions versus predictions of iso-electric point calculations.

\end{document}
